\usepackage{amsmath}
\usepackage{amscd}
\usepackage{latexsym}
\usepackage{amssymb}
\usepackage{amsfonts}
\usepackage{graphicx}
\usepackage{graphics}
\usepackage{makeidx}
%\makeindex[columns=2, title=\begin{center} Các ký hiệu dùng trong luận án \end{center}]
\makeindex
%\pagestyle{headings}
\topmargin=-1.5cm
\textwidth=15.5cm
\textheight=24cm
\headheight=2.5ex
\headsep=0.85cm
\oddsidemargin=1.cm
\evensidemargin=-.4cm
\parskip=0.7ex plus0.5ex minus 0.5ex
\baselineskip=17pt plus2pt minus2pt
%%%%%%%%%%%%%%%%%%%%%%%%%%%%%%%%%%%%%%%%%%%%%%%%%%%%%%%%%%%%%%%%%%%%%%%%%%%%%%%%%%%%%%%%
\catcode`@=11

%%%%%%%%%%%%%%%%%%%%%%%%%%%%%%%%%%%%%%%%%%%%%%%%%%%%%%%%%%%%%%%%%%%%%%%%%%%%%%%%%%%%%%%%%
\font\manual=manfnt scaled 1200
%%%%%%%%%%%%%%%%%%%%%%%%%%%%%%%%%%%%%%%%%%%%%%%%%%%%%%%%%%%%%%%%%%%%%%%%%%%%%%%%%%%%%%%%%
\def\blackbox{\rule{1ex}{0.25ex}\rule{0.25ex}{1.25ex}}
\def\vuongden{\makebox[0.52em]{\raisebox{-.07ex}{\blackbox}}}
\def\vuong{$\Box$\llap{\vuongden}}
\def\fordbend{{\manual\char127}}
\def\ford@nger{\smallbreak\begingroup\clubpenalty=10000
\def\par{\endgraf\endgroup\smallbreak}\noindent\hang\hangafter=-2
\hbox to1pt{\hskip-1.2\hangindent\fordbend\hfill}}
\outer\def\fordanger{\ford@nger}
\def\enddanger{\endgraf\endgroup}
\newcommand{\remark}{\par\noindent{{\sc Chú ý }.}\rm\,~}
\gdef\contentsname{Mục lục}
\gdef\lmdname{Mở đầu}
\gdef\lmd{\chapter*{%
\lmdname\markboth{\protect\it{\lmdname}}
{\protect\it{\lmdname}}}%
\addcontentsline{toc}{chapter}{\lmdname}}

\gdef\lctname{Lời cảm ơn}
\gdef\lct{\chapter*{%
\lctname\markboth{\protect\it{\lctname}}
{\protect\it{\lctname}}}%
}
%\addcontentsline{toc}{chapter}{\lctname}}

\gdef\lcdname{Lời cam đoan}
\gdef\lcd{\chapter*{%
\lcdname\markboth{\protect\it{\lcdname}}
{\protect\it{\lcdname}}}%
}
%\addcontentsline{toc}{chapter}{\lcdname}}

\gdef\lkname{Lời kết}
\gdef\lk{\chapter*{%
\lkname\markboth{\protect\it{\lkname}}
{\protect\it{\lkname}}}%
\addcontentsline{toc}{chapter}{\lkname}}

\gdef\chaptername{Chương}
\renewcommand\listfigurename{Danh sách hình}
\renewcommand\listtablename{Danh sách bảng}
\renewcommand\appendixname{Phụ lục}
\gdef\appendname{Phụ lục}
\gdef\append{\chapter*{%
\appendname\markboth{\protect\it{\appendname}}
{\protect\it{\appendname}}}%
\addcontentsline{toc}{chapter}{\appendname}}
\gdef\bibname{Tài liệu tham khảo}
\font\quan=cmbxti10 scaled 1200
%=============================
% cac dinh ly cho phan mo dau
%=============================
\newtheorem{dl}{Định lí}[chapter]
\newenvironment{dinhli}{\begin{dl}\rm}{\end{dl}}
\newtheorem{md}{Mệnh đề}[chapter]
\newenvironment{menhde}{\begin{md}\rm}{\end{md}}
\newtheorem{hq}{Hệ quả}[chapter]
\newenvironment{hequa}{\begin{hq}\rm}{\end{hq}}
\newtheorem{bd}{Bổ đề}[chapter]
\newenvironment{bode}{\begin{bd}\rm}{\end{bd}}
\newtheorem{td}{Tiên đề}
\newenvironment{tiende}{\begin{td}\rm}{\end{td}}
\theoremstyle{definition}
\newtheorem{kn}{Khái niệm}[chapter]
\newenvironment{khainiem}{\begin{kn}\rm}{\end{kn}}
\newtheorem{nx}{Nhận xét}[chapter]
\newenvironment{nxet}{\begin{nx}\rm}{\end{nx}}
\newtheorem{dn}{Định nghĩa}[chapter]
\newenvironment{dnghia}{\begin{dn}\rm}{\end{dn}}
\newtheorem{cy}{Chú ý}[chapter]
\newenvironment{chuy}{\begin{cy}\rm}{\end{cy}}
\newtheorem{vd}{Ví dụ}[chapter]
\newenvironment{vidu}{\begin{vd}\rm}{\end{vd}}
%\newenvironment{dinhly}{\begin{dl}\it}{\end{dl}}
%\newenvironment{dinhnghia}{\begin{dn}\rm}{\end{dn}}
\newtheorem{tc}{Tính chất}[chapter]
\newenvironment{tinhchat}{\begin{tc}\it}{\end{tc}}
%\newenvironment{menhde}{\begin{md}\it}{\end{md}}
%\newenvironment{vidu}{\begin{vd}\rm}{\hfill $\lhd$\nolinebreak\end{vd}}
%\newenvironment{baitap}{\begin{bta}\rm}{\nolinebreak\end{bta}}
%\newenvironment{bode}{\begin{bd}\it}{\end{bd}}
\newcommand{\cm}[1]{\bf\noindent{\noindent{Chứng minh}.~}%
\rm#1\hfill \rule{1ex}{1ex}\medbreak}
\newcommand{\cmdl}[2]{\bf\noindent{Chứng minh {#1}.~}%
\rm#2\hfill \rule{2.5mm}{2.5mm}\medbreak}
\newcommand{\cmbd}[2]{\bf\noindent{Chứng minh {#1}.~}%
\rm#2\hfill \rule{1ex}{1ex}\medbreak}
\newcommand{\solve}{\par\noindent{\sc Giải.}\,~}
%\newenvironment{hequa}{\begin{hq}\it}{\end{hq}}
\newcounter{lk}
\newenvironment{lietke}{\begin{list}{\rm(\roman{lk})}{\usecounter{lk}%
\setlength{\topsep}{0ex plus0.1ex}%
\setlength{\labelwidth}{1cm}%
\setlength{\itemsep}{0ex plus0.1ex}%
\setlength{\itemindent}{0.5cm}%
}}{\end{list}}
\newcounter{lka}
\newenvironment{lietkea}{\begin{list}{\rm(\alph{lka})}{\usecounter{lka}%
\setlength{\topsep}{0ex plus0.1ex}%
\setlength{\labelwidth}{1cm}%
\setlength{\itemsep}{0ex plus0.1ex}%
\setlength{\itemindent}{0.5cm}%
}}{\end{list}}
\renewcommand\indexname{Chỉ mục}
\gdef\introname{Giới thiệu}
\gdef\introduction{\chapter*{%
\introname\markboth{\protect\it{\introname}}
{\protect\it{\introname}}}%
\addcontentsline{toc}{chapter}{\introname}}
\newcounter{danhso}
\setcounter{danhso}{0}
\def\num{\addtocounter{danhso}{1}\par{\rm\arabic{danhso}.}\hskip.4em\relax}
\def\lnum{\addtocounter{danhso}{1}\par{\rm\arabic{danhso}\setcounter{danhso}{0}.}\hskip.4em\relax}
%%%%%%%%%%%%%%%%%%%%%%%%%%%%%%%%%%%%%%%%%%%%%%%%%%%%%%%%%%%%%%%%%%%%%%%%%%%%%
\setcounter{secnumdepth}{3}
\setcounter{tocdepth}{3}

\def\underarrow#1{\mathop{\setlength{\unitlength}{1cm}\begin{picture}(#1,0)\vector(1,0){#1}\end{picture}}}%
\def\realset{{I\!\! R}}%
\def\naturalset{{I\!\! N}}%
\def\therefore{\setlength{\unitlength}{2pt}%
\mbox{\raisebox{-0.5ex}{\begin{picture}(2,2)%
\put(2,0){$\cdot$}%
\put(0,0){$\cdot$}%
\put(1,2){$\cdot$}\end{picture}}\,}}%
\def\dautpmat#1#2{{$\displaystyle\int\limits_{#1}$\hskip#2}\llap{$\displaystyle\int$}}%
\def\surfoint#1#2#3{\mbox{\dautpmat{#1}{#2}\llap{$\bigcirc$\hskip#3}}}%
\def\hoitudeu{\mbox{%
\setlength{\unitlength}{1cm}%
\begin{picture}(0.7,0.5)%
\put(0,0.2){\vector(1,0){0.7}}%
\put(0,0.05){\vector(1,0){0.7}}%
\end{picture}}}%
\def\Tunit{\mbox{$\bf\hat{T}$}}%
\def\Bunit{\mbox{$\bf\hat{B}$}}%
\def\Nunit{\mbox{$\bf\hat{N}$}}%
\def\nacntoint{{$\circlearrowright$\hskip0.06cm}\llap{$\displaystyle\int$}}
\def\noint#1{\hbox{\nacntoint}_{\!\!\!#1}\,}
\def\pacntoint{{$\circlearrowleft$}\llap{$\displaystyle\int$}}
\def\point#1{\hbox{\pacntoint}_{\!\!\!#1}\,}
\def\nacntointl{{\small$\circlearrowright$}\llap{$\int$}}
\def\nointl#1{\hbox{\nacntointl}_{#1}\,}
\def\pacntointl{{\small$\circlearrowleft$\hskip-0.04cm}\llap{$\int$}}
\def\pointl#1{\hbox{\pacntointl}_{\,#1}\,}
\def\dauboihai{\mathop{\int\!\!\!\int}}
\def\tpboihai#1{\dauboihai\limits_{\hskip-0.6em#1}}
\def\dauboiba{\mathop{\int\!\!\!\int\!\!\!\int}}
\def\tpboiba#1{\dauboiba\limits_{\hskip-0.6em#1}}
\newcounter{lkl}
\newenvironment{lietkel}{\begin{list}{\rm[\arabic{lkl}]}{\usecounter{lkl}%
\setlength{\topsep}{0ex plus0.1ex}%
\setlength{\labelwidth}{1cm}%
\setlength{\itemsep}{0ex plus0.1ex}%
\setlength{\itemindent}{0cm}%
}}{\end{list}}
%%%%%%%%%%%%%%%%%%%%%%%%%%%%%%%%%%%%%%%%%%%%%%%%%%%%%%%%%%%%%%%%%%%%%%%%%%%%%%%%%%%%%%%%%%%%%%
\newcounter{donvihoanh}
\newcounter{donvitung}
\newbox\unitnum
\gdef\vetruchoanh#1#2#3#4#5{%
\setlength{\unitlength}{#1pt}
\count1=#2\count2=#2
\multiply\count1 by 10
\advance\count2 by 1
\put(0,0){\vector(1,0){\count2}}
{\linethickness{0.1pt}\multiput(0,0)(0.1,0){\count1}{\setlength{\unitlength}{10pt}\line(0,-1){0.3}}}
\multiput(1,0)(1,0){#2}{\setlength{\unitlength}{10pt}\line(0,-1){0.6}}
\setcounter{donvihoanh}{#3}
\multiput(0,0)(1,0){\count2}{\ifnum\arabic{donvihoanh}=0
\tiny\kern-0.5ex\raise-5ex\hbox{\arabic{donvihoanh}}\else
\tiny\setbox\unitnum=\hbox{\arabic{donvihoanh}}
\hspace{-0.72\wd\unitnum}\raise-5ex\hbox{\arabic{donvihoanh}}\fi\addtocounter{donvihoanh}{#4}}
\put(\count2,0){\tiny\raise-5ex\hbox{\footnotesize$#5$}}
}
\gdef\vetructung#1#2#3#4#5{%
\setlength{\unitlength}{#1pt}
\count1=#2\count2=#2
\multiply\count1 by 10
\advance\count2 by 1
\put(0,0){\vector(0,1){\count2}}
{\linethickness{0.1pt}\multiput(0,0)(0,0.1){\count1}{\setlength{\unitlength}{10pt}\line(-1,0){0.3}}}
\multiput(0,1)(0,1){#2}{\setlength{\unitlength}{10pt}\line(-1,0){0.6}}
\setcounter{donvitung}{#3}
\multiput(0,0)(0,1){\count2}{\tiny\setbox\unitnum=\hbox{\arabic{donvitung}}
\kern-5.5ex\hspace{-0.65\wd\unitnum}\raise-0.85ex\hbox{\arabic{donvitung}}\addtocounter{donvitung}{#4}}
\put(0,\count2){\kern-2.5ex\tiny\raise-1ex\hbox{\footnotesize$#5$}}
}
%#1: don vi duoc chon tren truc (tinh bang so pt).
%#2: so khoang can bieu dien tren truc.
%#3: vi tri diem bieu dien nho nhat tren truc.
%#4: so don vi cua moi khoang.
%#5: ten bien tren truc.
%%%%%%%%%%%%%%%%%%%%%%%%%%%%%%%%%%%%%%%%%%%%%%%%%%%%%%%%%%%%%%%%%%%%%%%%%%%%%%%%%%%%%%
