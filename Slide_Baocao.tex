\documentclass[notheorems,envcountsect,aspectratio=169,unicode]{beamer}
\usefonttheme{serif}

% Các gói hỗ trợ tiếng Việt và định dạng cơ bản
\usepackage[utf8]{inputenc}
\usepackage[T5]{fontenc} % Cho tiếng Việt
\usepackage{graphicx} % Hình ảnh
\usepackage{amsmath, amssymb} % Toán học nâng cao
\usepackage{mathrsfs} % Phông chữ toán học đặc biệt
\usepackage{float} % Điều chỉnh vị trí hình ảnh
\usepackage{xcolor} % Màu sắc
\usepackage{tcolorbox} % Hộp màu (nếu cần)

% Khai báo màu sắc
\definecolor{dkgreen}{rgb}{0,0.6,0}
\definecolor{gray}{rgb}{0.5,0.5,0.5}
\definecolor{mauve}{rgb}{0.58,0,0.82}
\definecolor{LightGray}{gray}{0.9}
\setbeamercolor{uppercol}{fg=white,bg=blue}
\setbeamercolor{lowercol}{fg=black,bg=white}%

% Tùy chỉnh giao diện Beamer
\mode<presentation> {
	\usetheme{default}
	\usecolortheme[named=black]{structure} % Màu đen cho tiêu đề
	\setbeamercolor{normal text}{fg=black} % Văn bản màu đen
	\setbeamercolor{background canvas}{bg=white} % Nền trắng
	\setbeamertemplate{itemize item}{\color{black}$\bullet$} % Ký hiệu danh sách
	\setbeamertemplate{navigation symbols}{} % Ẩn thanh điều hướng
	\setbeamertemplate{footline} % Hiển thị số trang dưới cùng
	{\begin{minipage}{5mm} \vspace{-10 mm} \hfill \insertframenumber \end{minipage}}
}


%%%%%%%%%%%%%%%%%%%%%%%%%%%%%%%%
\newtcbox{\xmybox}[1][red]{on line,
	arc=7pt,colback=#1!10!white,colframe=#1!50!black,
	before upper={\rule[-3pt]{0pt}{10pt}},boxrule=1pt,
	boxsep=0pt,left=4pt,right=4pt,top=2pt,bottom=2pt}
%%%%%%%%%%%
\newtcbox{\xmyboxx}[1][blue]{on line,
	arc=7pt,colback=#1!10!white,colframe=#1!50!black,
	before upper={\rule[-3pt]{0pt}{10pt}},boxrule=1pt,
	boxsep=0pt,left=4pt,right=4pt,top=2pt,bottom=2pt}
%%%%%%%%%%%
\newtcbox{\xmyboxy}[1][green]{on line,
	arc=7pt,colback=#1!10!white,colframe=#1!50!black,
	before upper={\rule[-3pt]{0pt}{10pt}},boxrule=1pt,
	boxsep=0pt,left=4pt,right=4pt,top=2pt,bottom=2pt}
% Bắt đầu tài liệu
\begin{document}

% Trang tiêu đề
\setbeamertemplate{background}{\includegraphics[width=\paperwidth]{CTU_Title.png}} % Hình nền trang tiêu đề
\begin{frame}[plain]
	\vspace*{12ex}
\centering\textcolor{white}{\textbf{\normalsize BÁO CÁO LUẬN VĂN TỐT NGHIỆP}}\\

\vspace*{2ex}

    \centering   \textcolor{yellow}{\textbf{PHƯƠNG PHÁP BÌNH PHƯƠNG TỐI TIỂU \\ VÀ BÀI TOÁN CÂN BẰNG TRONG VẬT LÝ}}

  \vspace*{5ex}  
\centering\textcolor{white}{\textbf{Sinh viên thực hiện: Đỗ Mỹ Duyên}}
  
 \centering \textcolor{white}{\textbf{Cán bộ hướng dẫn: TS. Phạm Bích Như}}

 \vspace*{10ex} 
 \centering \textcolor{white}{\textbf{Cần Thơ, ngày 30.4.2025}}
\end{frame}

% Các slide nội dung
\setbeamertemplate{background}{\includegraphics[width=\paperwidth]{CTU_Body.png}} % Hình nền các slide nội dung
\setbeamertemplate{frametitle}{
	\nointerlineskip%
	\begin{beamercolorbox}[wd=\paperwidth,ht=4.0ex,dp=1.6ex]{frametitle}
		\hspace*{8ex}\insertframetitle
	\end{beamercolorbox}
}

	%----------------------------------------------------------------------------------------
	%	PRESENTATION SLIDES
	\fontsize{14 pt}{20 pt}\selectfont
	\begin{frame}
		\frametitle{\textcolor{violet}{\begin{center}
					\fontsize{20pt}{30pt} \textbf{\vspace*{-1ex} Mệnh đề và Chân trị}
		\end{center}} }
		\transblindshorizontal \vspace{-3ex}
	\begin{enumerate}
    \item Các khái niệm cơ bản về mệnh đề
    \begin{itemize}
        \item Định nghĩa mệnh đề
        \item Giá trị chân lý (chân trị) (Đúng/Sai)
        \item Ví dụ 
        \end{itemize}    
     \item Mệnh đề phức
    \begin{itemize}
        \item Phép hội ($\wedge$)
        \item Phép tuyển ($\vee$)
        \item Phép phủ định ($\neg$)
        \item Bảng giá trị đúng sai
    \end{itemize}
    \end{enumerate}
	\end{frame}	

   %%%%%%%%%%%%%%%%%

    %%%%%%%%%%%%%%%%%%%%%%%%%%%%%%%%%%%%%%%%
       \fontsize{14pt}{20 pt}\selectfont
	\begin{frame}
        \frametitle{\textcolor{violet}{\begin{center} \fontsize{18pt}{30pt}\textbf{\vspace*{-1ex} Nội dung} \end{center}}}
		\transblindshorizontal \vspace{-3ex}

     
	\end{frame}	
    %%%%%%%%%%%%%%%%%%%%%%%%%%%%%%%%%%%%%%%
  
    %%%%%%%%%%%%%%%%%%%%%%%%%%%%%%%%%%%%%%%%
    \fontsize{16 pt}{20 pt}\selectfont
\begin{frame}	\frametitle{\textcolor{violet}{\begin{center}
				\fontsize{18pt}{30pt} \textbf{\vspace*{-1ex} Bình phương tối tiểu}
	\end{center}} }
\transblindshorizontal 
\vspace{-4ex}
\begin{minipage}[b]{14cm}
		\begin{beamerboxesrounded}[upper=uppercol,lower=lowercol,shadow=true]{Định nghĩa 1}
 NỘI DUNG
\end{beamerboxesrounded}
	\end{minipage}
\vspace{-0.5ex}	
\end{frame}	
%%%%%%%%%%%%%%%%%%%%
 \fontsize{16 pt}{20 pt}\selectfont
\begin{frame}	\frametitle{\textcolor{violet}{\begin{center}
				\fontsize{18pt}{30pt} \textbf{\vspace*{-1ex} Bình phương tối tiểu}
	\end{center}} }
\transblindshorizontal 
\vspace{-2ex}
\begin{minipage}[b]{14cm}
		\begin{beamerboxesrounded}[upper=uppercol,lower=lowercol,shadow=true]{Định lý 2.1}
 
\end{beamerboxesrounded}
	\end{minipage}
\vspace{-0.5ex}	
\end{frame}	
 
   %%%%%%%%%%%%%%%%%%%%%%%%%%%%%%%%%%%%%%%%%%
   \fontsize{14pt}{20 pt}\selectfont
	\begin{frame}
        \frametitle{\textcolor{violet}{\begin{center} \fontsize{18pt}{30pt}\textbf{\vspace*{-1ex} Tài liệu tham khảo}\end{center}}}
		\transblindshorizontal \vspace{-3ex}
\begin{itemize}
\item[1.] 
\item[2.] 
\item[3.]           
          \end{itemize} 
	\end{frame}	
   %%%%%%%%%%%%%%%%%%%%%%%%%%%%%%%%%%%%% 
  
   %%%%%%%%%%%%%%%%%%%%%%%%%%%%%%%%%%%%%KẾT LUẬN
	\setbeamertemplate{background}{\includegraphics[width=\paperwidth]{CTU_Clusion.png}} % Adding the backgroundlogo for the rest of the slides
	\setbeamertemplate{frametitle}{
		\nointerlineskip%
		\begin{beamercolorbox}[wd=\paperwidth,ht=4.0ex,dp=1.6ex]{frametitle}
			\hspace*{10ex}\insertframetitle%
		\end{beamercolorbox}
	}
\begin{frame}

\end{frame}

\end{document} 