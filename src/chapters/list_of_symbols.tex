\chapter*{DANH SÁCH KÝ HIỆU VÀ VIẾT TẮT}
\addcontentsline{toc}{chapter}{DANH SÁCH KÝ HIỆU VÀ VIẾT TẮT}

\section*{Ký hiệu toán học cơ bản}
\begin{longtable}{@{}p{3cm}p{10cm}@{}}
\textbf{Ký hiệu} & \textbf{Ý nghĩa} \\
\hline
\endhead
$\Omega$ & Không gian mẫu \\
$\mathcal{F}$ & $\sigma$-đại số các biến cố \\
$P$ & Độ đo xác suất \\
$X, Y, Z$ & Biến ngẫu nhiên \\
$\mathbf{X}$ & Vector ngẫu nhiên \\
$x, y, z$ & Giá trị quan sát của biến ngẫu nhiên \\
$f_X(x)$ & Hàm mật độ xác suất (PDF) \\
$F_X(x)$ & Hàm phân phối tích lũy (CDF) \\
$F_n(x)$ & Hàm phân phối thực nghiệm (ECDF) \\
$\mathbb{E}[X]$ & Kỳ vọng của biến ngẫu nhiên $X$ \\
$\text{Var}(X)$ & Phương sai của biến ngẫu nhiên $X$ \\
$\text{Cov}(X,Y)$ & Hiệp phương sai giữa $X$ và $Y$ \\
$\rho_{XY}$ & Hệ số tương quan giữa $X$ và $Y$ \\
$\sigma$ & Độ lệch chuẩn tổng thể \\
$\sigma^2$ & Phương sai tổng thể \\
$\mu$ & Trung bình tổng thể \\
$\boldsymbol{\mu}$ & Vector kỳ vọng \\
$\boldsymbol{\Sigma}$ & Ma trận hiệp phương sai \\
$\mathbf{R}$ & Ma trận tương quan \\
$n$ & Kích thước mẫu \\
$\overline{X}$ & Trung bình mẫu \\
$S^2$ & Phương sai mẫu \\
$S$ & Độ lệch chuẩn mẫu \\
\end{longtable}

\section*{Ký hiệu phân phối xác suất}
\begin{longtable}{@{}p{3cm}p{10cm}@{}}
\textbf{Ký hiệu} & \textbf{Ý nghĩa} \\
\hline
\endhead
$\mathcal{N}(\mu,\sigma^2)$ & Phân phối chuẩn với trung bình $\mu$ và phương sai $\sigma^2$ \\
$\mathcal{N}_p(\boldsymbol{\mu},\boldsymbol{\Sigma})$ & Phân phối chuẩn nhiều chiều \\
$\chi^2(n)$ & Phân phối Chi-bình phương với $n$ bậc tự do \\
$t(n)$ & Phân phối Student với $n$ bậc tự do \\
$F(n_1,n_2)$ & Phân phối F với bậc tự do $(n_1,n_2)$ \\
$\text{B}(n,p)$ & Phân phối nhị thức với tham số $n$ và $p$ \\
$\text{Poisson}(\lambda)$ & Phân phối Poisson với tham số $\lambda$ \\
$U(a,b)$ & Phân phối đều trên đoạn $[a,b]$ \\
$\text{Exp}(\lambda)$ & Phân phối mũ với tham số $\lambda$ \\
$\text{Gamma}(\alpha,\beta)$ & Phân phối Gamma với tham số hình dạng $\alpha$ và tỷ lệ $\beta$ \\
\end{longtable}

\section*{Ký hiệu kiểm định thống kê}
\begin{longtable}{@{}p{3cm}p{10cm}@{}}
\textbf{Ký hiệu} & \textbf{Ý nghĩa} \\
\hline
\endhead
$H_0$ & Giả thuyết không (null hypothesis) \\
$H_1$ & Giả thuyết đối (alternative hypothesis) \\
$\alpha$ & Mức ý nghĩa (significance level) \\
$\beta$ & Xác suất sai lầm loại II \\
$1-\beta$ & Lực kiểm định (power of test) \\
$p$-value & Giá trị $p$ (probability value) \\
$\chi^2_{\text{obs}}$ & Giá trị quan sát của thống kê Chi-bình phương \\
$D_n$ & Thống kê Kolmogorov-Smirnov \\
$O_i$ & Tần số quan sát (observed frequency) \\
$E_i$ & Tần số kỳ vọng (expected frequency) \\
$z_{\alpha}$ & Phân vị chuẩn tại mức $\alpha$ \\
$t_{\alpha,n}$ & Phân vị Student tại mức $\alpha$ với $n$ bậc tự do \\
$\chi^2_{\alpha,n}$ & Phân vị Chi-bình phương tại mức $\alpha$ với $n$ bậc tự do \\
\end{longtable}

\section*{Ký hiệu phân tích nhiều chiều}
\begin{longtable}{@{}p{3cm}p{10cm}@{}}
\textbf{Ký hiệu} & \textbf{Ý nghĩa} \\
\hline
\endhead
$\mathbf{A}$ & Ma trận các vector riêng trong PCA \\
$\boldsymbol{\Lambda}$ & Ma trận chéo các trị riêng \\
$\lambda_i$ & Trị riêng thứ $i$ \\
$\mathbf{a}_i$ & Vector riêng thứ $i$ \\
$Y_i$ & Thành phần chính thứ $i$ \\
$\mathbf{L}$ & Ma trận tải nhân tố (factor loading matrix) \\
$\mathbf{F}$ & Vector nhân tố chung \\
$\boldsymbol{\epsilon}$ & Vector sai số cụ thể \\
$\boldsymbol{\Psi}$ & Ma trận phương sai cụ thể \\
$h_i^2$ & Phương sai chung (communality) của biến thứ $i$ \\
$\psi_i$ & Phương sai cụ thể (specific variance) của biến thứ $i$ \\
$KMO$ & Chỉ số Kaiser-Meyer-Olkin \\
$MSA$ & Measure of Sampling Adequacy \\
\end{longtable}

\section*{Ký hiệu hội tụ và giới hạn}
\begin{longtable}{@{}p{3cm}p{10cm}@{}}
\textbf{Ký hiệu} & \textbf{Ý nghĩa} \\
\hline
\endhead
$\xrightarrow{P}$ & Hội tụ theo xác suất \\
$\xrightarrow{a.s.}$ & Hội tụ hầu chắc chắn \\
$\xrightarrow{\mathcal{D}}$ & Hội tụ theo phân phối \\
$\Rightarrow$ & Hội tụ yếu (weak convergence) \\
$\overset{d}{\approx}$ & Xấp xỉ theo phân phối \\
$\overset{i.i.d.}{\sim}$ & Độc lập cùng phân phối \\
$\mathbf{1}_A$ & Hàm chỉ báo của tập $A$ \\
$\sup$ & Supremum (cận trên nhỏ nhất) \\
$\inf$ & Infimum (cận dưới lớn nhất) \\
\end{longtable}

\section*{Viết tắt và thuật ngữ}
\begin{longtable}{@{}p{4cm}p{9cm}@{}}
\textbf{Viết tắt} & \textbf{Ý nghĩa} \\
\hline
\endhead
\textbf{Lý thuyết xác suất} & \\
LTXS & Lý thuyết xác suất \\
TKTH & Thống kê toán học \\
CLT & Central Limit Theorem (Định lý giới hạn trung tâm) \\
LLN & Law of Large Numbers (Luật số lớn) \\
CDF & Cumulative Distribution Function (Hàm phân phối tích lũy) \\
PDF & Probability Density Function (Hàm mật độ xác suất) \\
PMF & Probability Mass Function (Hàm khối xác suất) \\
ECDF & Empirical Cumulative Distribution Function \\
\\
\textbf{Kiểm định thống kê} & \\
GOF & Goodness of Fit (Kiểm định tính phù hợp) \\
K-S & Kolmogorov-Smirnov \\
MRF & Markov Random Field \\
\\
\textbf{Phân tích nhiều chiều} & \\
PCA & Principal Component Analysis (Phân tích thành phần chính) \\
FA & Factor Analysis (Phân tích nhân tố) \\
EFA & Exploratory Factor Analysis (Phân tích nhân tố khám phá) \\
CFA & Confirmatory Factor Analysis (Phân tích nhân tố khẳng định) \\
KMO & Kaiser-Meyer-Olkin \\
MSA & Measure of Sampling Adequacy \\
ML & Maximum Likelihood (Cực đại hóa likelihood) \\
\\
\textbf{Công cụ và phần mềm} & \\
MATLAB & Matrix Laboratory \\
SPSS & Statistical Package for the Social Sciences \\
SAS & Statistical Analysis System \\
\\
\textbf{Tổ chức} & \\
CTU & Can Tho University (Trường Đại học Cần Thơ) \\
ĐHCT & Đại học Cần Thơ \\
KHTN & Khoa Học Tự Nhiên \\
\end{longtable}
