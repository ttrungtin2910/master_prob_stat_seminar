%HƯỚNG DẪN CÁCH GHI TLTK XÓA BỎ KHI HOÀN THANH LUẬN VĂN
\begin{longtable}{|p{3cm}|p{5cm}|p{5cm}|}
    \hline
      Loại trích dẫn   & Trích dẫn trong ngoặc đơn (Parenthetical citation) & Trích dẫn trong câu 
(Narrative citation)
\\
     \hline
       \textbf{Một tác giả}
       
Ghi tác giả và năm
  & (Hường, 2013)
  
(Tain, 1999)
 & Hường (2013)
 
Tain (1999)
\\
    \hline
   \textbf{Hai tác giả}
   
Ghi hai tác giả và năm		
      & (Deharveng $\&$ Bedos, 2000) 
      
(Hồ $\&$ Lư, 2003) & Deharveng and Bedos (2000)

Hồ và Lư (2003)\\
         \hline
       \textbf{Ba tác giả trở lên}
       
Ghi tác giả đầu tiên, theo sau là ``và ctv.'' hoặc ``et al.'' và năm 
  & (Aron et al., 2019)
  
(Hiền và ctv., 2016)

*``và ctv.'', ``et al.'' không viết in nghiêng
 & Aron et al. (2019)
 
Hiền và ctv. (2016)
\\
\hline
  \textbf{Tác giả là một cơ quan, tổ chức}
  
Ghi tên cơ quan và năm (Tên cơ quan có thể viết tắt nếu được trích dẫn hơn một lần trong bài)
       &  (United States Government Accountability Office, 2019)

*Trích dẫn lần đầu:
 (Food and Agriculture Organization of the United Nations [FAO], 1977)
 
*Trích dẫn lần sau:
(FAO, 1977)
 & United States Government Accountability Office (2019)

*Trích dẫn lần đầu:
Food and Agriculture Organization of the United Nations (FAO, 2020)

*Trích dẫn lần sau:
FAO (1977)
 \\
      \hline
       \textbf{Nhiều tài liệu}
       
Sắp xếp các tài liệu theo năm xuất bản tăng dần. Nếu các tài liệu có cùng năm xuất bản, thì sắp xếp theo thứ tự bảng chữ cái.
  & (Hồng và ctv. 2014; Hiền và ctv., 2016; Bộ Giáo dục và Đào tạo, 2017; Cảnh, 2017; Aron, 2019; Belcher, 2019)
  
  *Mỗi tài liệu cách nhau bằng dấu chấm phẩy 
  & Hồng và ctv. (2014), Hiền và ctv. (2016), Bộ Giáo dục và Đào tạo (2017), Cảnh 92017), Aron (2019) và Belcher (2019) \\
         \hline
     \textbf{Nhiều tài liệu cùng cách trích dẫn tác giả}
     
Ghi tác giả và các năm theo thứ tự tăng dần
    & (Vuong et al., 2018, 2019b)

(Cảnh, 2017, 2020)
 & Vuong et al. (2018, 2019b)

Cảnh (2017, 2020)
 \\
         \hline
  \textbf{Nhiều tài liệu cùng cách trích dẫn tác giả và cùng năm xuất bản}
  
Ghi tác giả và năm kèm theo chữ cái a, b, c,$\cdots$ & (Vuong et al., 2019a, 2019b)

(Thanh và ctv., 2021a, 2021b)
 & Vuong et al. (2019a, 2019b)

Thanh và ctv. (2021a, 2021b)
\\
         \hline
      \textbf{Trích dẫn từ nguồn thứ cấp}
      
Ghi tác giả và năm (nếu có) của tài liệu gốc kèm "trích dẫn bởi" hoặc "as cited in" tác giả và năm của tài liệu thứ cấp
   & (Garrison, 2011, as cited in Kattoua et al., 2016)

(Hinh và ctv., 2003, trích dẫn bởi Tuấn $\&$ Minh, 2015)

*Trong danh mục TLTK chỉ liệt kê tài liệu thứ cấp (Kattoua et al., 2016; Tuấn $\&$ Minh, 2015)
 & Garrison (2011, as cited in Kattoua et al., 2016) 

Hinh và ctv. (2013, trích dẫn bởi Tuấn $\&$ Minh, 2015)
\\
        \hline
\end{longtable}
%%%%%%%%%%%%%%%%%%%%%%%%%%%%%%%%%%%%%%%%%%%%%%%%%
\begin{longtable}{|p{3cm}|p{5cm}|p{5cm}|}
    \hline
      Loại trích dẫn   & Trích dẫn trong ngoặc đơn (Parenthetical citation) & Trích dẫn trong câu 
(Narrative citation)
\\
     \hline       
     \textbf{Trích dẫn nguyên văn}
     
Ghi tác giả, năm và trang viết.


Đoạn trích dưới 40 từ: để trong ngoặc kép.
Đoạn trích trên 40 từ: viết riêng đoạn mới, lùi đầu dòng, không dấu ngoặc kép.
    & ``Riêng hai tiếng Cần Thơ trong sử không có ghi chép rõ ràng như các tỉnh khác'' (Minh, 1966, trích dẫn bởi Cảnh, 2020, tr. 232). 


Nguồn gốc tên gọi ``Cần Thơ'' do dân gian truyền lại như sau:

Tương truyền lúc chúa Nguyễn Ánh trên đường bôn tẩu vào Nam đã đi qua nhiều nơi để tránh Tây Sơn mưu đồ phục quốc. Lúc bấy giờ Ngài ngự trên một chiếc thuyền đi ngang dòng sông Hậu, thuộc địa phận huyện Phong Phú thả thuyền theo sóng gió lênh đênh trên mặt nước, bỗng nghe tiếng ngâm thơ, đàn địch, hò hát, hòa nhau rất nhịp nhàng. Ngài xúc động và đặt tên con sông này là Cầm Thi giang. Lần lần hai tiếng Cầm Thi được lan rộng ra dân chúng, được đọc trại là "Cần Thơ". (Minh, 1966, trích dẫn bởi Cảnh, 2020, tr. 232)
 & Trong sách Cần Thơ xưa và nay, soạn giả Minh (1966, trích dẫn bởi Cảnh, 2020) cũng cho rằng: "Riêng hai tiếng "Cần Thơ" trong sử không có ghi chép rõ ràng như các tỉnh khác" (tr. 232). 

Minh (1966, trích dẫn bởi Cảnh, 2020) đã đề cập đến nguồn gốc tên gọi "Cần Thơ" do dân gian truyền lại như sau:

Tương truyền lúc chúa Nguyễn Ánh trên đường bôn tẩu vào Nam đã đi qua nhiều nơi để tránh Tây Sơn mưu đồ phục quốc. Lúc bấy giờ Ngài ngự trên một chiếc thuyền đi ngang dòng sông Hậu, thuộc địa phận huyện Phong Phú thả thuyền theo sóng gió lênh đênh trên mặt nước, bỗng nghe tiếng ngâm thơ, đàn địch, hò hát, hòa nhau rất nhịp nhàng. Ngài xúc động và đặt tên con sông này là Cầm Thi giang. Lần lần hai tiếng Cầm Thi được lan rộng ra dân chúng, được đọc trại là ``Cần Thơ''. (tr. 232)
\\
         \hline
\end{longtable}