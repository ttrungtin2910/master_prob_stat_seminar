\chapter*{Mở đầu}
\addcontentsline{toc}{chapter}{\textbf{MỞ ĐẦU}}

\section*{1. Lý do chọn đề tài}

Trong thời đại phát triển mạnh mẽ của khoa học dữ liệu và trí tuệ nhân tạo, thống kê học đóng vai trò nền tảng quan trọng trong việc phân tích, xử lý và rút ra những kết luận có ý nghĩa từ dữ liệu. Đặc biệt, thống kê nâng cao cung cấp những công cụ mạnh mẽ để giải quyết các bài toán phức tạp trong nghiên cứu khoa học, kinh tế, y học, và nhiều lĩnh vực khác.

Chuyên đề "Thống kê nâng cao" được lựa chọn nhằm trang bị cho người học những kiến thức sâu rộng về các phương pháp thống kê hiện đại, từ những kiến thức cơ bản đến các kỹ thuật phân tích tiên tiến. Việc nắm vững các phương pháp kiểm định giả thuyết, phân tích nhiều chiều và các kỹ thuật thống kê ứng dụng sẽ giúp người học có được nền tảng vững chắc để áp dụng vào thực tiễn nghiên cứu và công việc.

Hơn nữa, với sự bùng nổ của dữ liệu lớn (Big Data) và nhu cầu ngày càng cao về khả năng phân tích định lượng trong các ngành nghề, việc hiểu sâu về thống kê nâng cao trở thành một yêu cầu cần thiết đối với sinh viên ngành Toán - Thống kê và các ngành liên quan.

\section*{2. Mục tiêu và phạm vi nghiên cứu}

\subsection*{2.1. Mục tiêu nghiên cứu}

Mục tiêu chính của bài thu hoạch này nhằm:

\begin{itemize}
    \item \textbf{Mục tiêu tổng quát}: Tổng hợp và trình bày một cách có hệ thống các kiến thức cốt lõi về thống kê nâng cao, từ lý thuyết đến ứng dụng thực tiễn.
    
    \item \textbf{Mục tiêu cụ thể}:
    \begin{itemize}
        \item Hệ thống hóa các kiến thức nền tảng về lý thuyết xác suất và thống kê toán học
        \item Trình bày chi tiết các phương pháp kiểm định thống kê quan trọng như kiểm định Pearson, Kolmogorov-Smirnov và các dạng kiểm định khác
        \item Phân tích sâu các phương pháp phân tích dữ liệu nhiều chiều bao gồm phân tích thành phần chính (PCA), phân tích nhân tố, phân tích tương quan và hồi quy đa biến
        \item Cung cấp các ví dụ minh họa và ứng dụng thực tiễn sử dụng phần mềm MATLAB
        \item Xây dựng nền tảng lý thuyết vững chắc cho các nghiên cứu ứng dụng trong tương lai
    \end{itemize}
\end{itemize}

\subsection*{2.2. Phạm vi nghiên cứu}

Nội dung của bài thu hoạch được giới hạn trong các chủ đề chính sau:

\begin{itemize}
    \item \textbf{Về mặt lý thuyết}: Tập trung vào các khái niệm và định lý cơ bản của lý thuyết xác suất, các phân phối xác suất quan trọng, định lý giới hạn trung tâm và các nguyên lý cơ bản của suy diễn thống kê.
    
    \item \textbf{Về phương pháp kiểm định}: Nghiên cứu sâu về kiểm định tính phù hợp (goodness-of-fit), kiểm định độc lập, kiểm định so sánh phân phối và các dạng kiểm định phi tham số.
    
    \item \textbf{Về phân tích nhiều chiều}: Bao gồm các kỹ thuật giảm chiều dữ liệu, phân tích cụm, phân loại và các phương pháp học có giám sát cơ bản.
    
    \item \textbf{Về công cụ tính toán}: Sử dụng chủ yếu MATLAB làm công cụ minh họa và thực hành, với các đoạn code mẫu và ví dụ cụ thể.
\end{itemize}

\section*{3. Phương pháp nghiên cứu}

Để đạt được các mục tiêu đề ra, bài thu hoạch sử dụng phương pháp nghiên cứu tổng hợp kết hợp nhiều hướng tiếp cận:

\begin{itemize}
    \item \textbf{Phương pháp nghiên cứu tài liệu}: Tổng hợp và phân tích các tài liệu tham khảo từ các giáo trình kinh điển về thống kê toán học, các bài báo khoa học và tài liệu nghiên cứu chuyên sâu.
    
    \item \textbf{Phương pháp diễn giải và trình bày hệ thống}: Sắp xếp các kiến thức theo trình tự logic từ cơ bản đến nâng cao, từ lý thuyết đến ứng dụng.
    
    \item \textbf{Phương pháp minh họa bằng ví dụ}: Sử dụng các ví dụ cụ thể, bài toán thực tế để làm rõ các khái niệm và phương pháp lý thuyết.
    
    \item \textbf{Phương pháp thực nghiệm tính toán}: Xây dựng và chạy các chương trình MATLAB để minh họa các thuật toán và kiểm định các kết quả lý thuyết.
    
    \item \textbf{Phương pháp so sánh và phân tích}: So sánh ưu nhược điểm của các phương pháp khác nhau, phân tích điều kiện áp dụng và hiệu quả của từng kỹ thuật.
\end{itemize}

\section*{4. Cấu trúc của bài thu hoạch}

Bài thu hoạch được tổ chức thành các phần chính như sau:

\textbf{Chương 1: Kiến thức chuẩn bị}
\begin{itemize}
    \item Hệ thống lại các khái niệm cơ bản về không gian xác suất và biến ngẫu nhiên
    \item Trình bày các phân phối xác suất quan trọng và tính chất của chúng
    \item Định lý giới hạn trung tâm và luật số lớn
    \item Cơ sở của kiểm định giả thuyết và suy diễn thống kê
\end{itemize}

\textbf{Chương 2: Một số dạng kiểm định thống kê}
\begin{itemize}
    \item Kiểm định Pearson chi-bình phương cho tính phù hợp và độc lập
    \item Kiểm định Kolmogorov-Smirnov cho phân phối liên tục
    \item Các kiểm định phi tham số khác
    \item Ứng dụng và minh họa bằng MATLAB
\end{itemize}

\textbf{Chương 3: Phân tích nhiều chiều dữ liệu thang đo định lượng}
\begin{itemize}
    \item Phân tích thành phần chính (Principal Component Analysis)
    \item Phân tích nhân tố (Factor Analysis)
    \item Phân tích tương quan và hồi quy đa biến
    \item Các kỹ thuật giảm chiều và trực quan hóa dữ liệu
    \item Ứng dụng trong xử lý dữ liệu thực tế
\end{itemize}

\textbf{Kết luận}
\begin{itemize}
    \item Tổng kết các kiến thức đã trình bày
    \item Đánh giá ý nghĩa và ứng dụng của thống kê nâng cao
    \item Hướng phát triển và nghiên cứu tiếp theo
\end{itemize}

Mỗi chương đều được thiết kế với cấu trúc rõ ràng, bao gồm nền tảng lý thuyết vững chắc, các định lý và công thức quan trọng, ví dụ minh họa cụ thể và ứng dụng thực tiễn. Điều này giúp người đọc có thể nắm bắt được cả kiến thức lý thuyết sâu sắc và khả năng ứng dụng thực tế của các phương pháp thống kê nâng cao.
