\chapter{Kiến thức chuẩn bị}
\setcounter{page}{1}
\pagenumbering{arabic}
Nội dung này được tham khảo và tổng hợp từ các tài liệu tham khảo \cite{1}, \cite{2} và \cite{5}. 


\section{Mục 1 không có dấu chấm cuối dòng}
\subsection{Tiểu mục 1 không có dấu chấm cuối dòng}

Mỗi đoạn diễn đạt cho mỗi ý bao gồm nhiều câu. Mỗi câu phải đầy đủ cú pháp: S + V + O (nếu có).

Qua đoạn mới phải thụt đầu dòng bằng cách chừa 1 dòng trống trong Latex. Thông thường người ta rất hạn chế sử dụng $\setminus\setminus$ để xuống dòng. 
\begin{definition}
 Nội dung định nghĩa viết vào đây
\end{definition}

\[ \int_{x=0}^5 f_X (x) {\rm d}x=1 \]

\begin{definition}
 Nội dung định nghĩa viết vào đây có thể sửa lại
\end{definition}

\begin{theorem}
 Nội dung định lý viết vào đây   
\end{theorem}

\begin{proof}
Nội dung CM viết vào đây.
\end{proof}

\subsection{Tiểu mục 2}
%%%%%%%%%Chèn bảng

\begin{table}[ht]
 \caption{\textbf{Bảng minh họa 1}}
    \centering
    \begin{tabular}{|c|c|c|c|c|}
    \hline
     STT    & Họ và tên & Kiểm tra & Thi & Tổng\\
      \hline
     1    & Nguyễn Văn A &  &  & \\
      \hline
      2   &  &  &  & \\
       \hline
         &  &  &  & \\
       \hline
    \end{tabular} 
    %\label{tab:my_label}
\end{table}

%%%%%%%%%%%%%Dạng tùy chỉnh kích thước cột
\begin{table}[ht]
 \caption{\textbf{Bảng minh họa 2}}
 \centering
\begin{tabular}{|c|l|l|l|c|c|}
		\hline
		\textbf{Stt} &\textbf{Tên bài báo} & \textbf{Tác giả/nhóm tác giả}
		& \textbf{Tên tạp chí} &\textbf{Số tạp chí} & \textbf{Năm xuất bản}
		\\
		\hline
		1 &   &  & &  & \\
		\hline
	2 &   &  & &  & \\
		\hline
        3 &   &  & &  & \\
		\hline
	\end{tabular}
    \end{table}
%%%%%%%%%%%%%%%%%%%%%%%%%%%%%%%%%%%%%%%%%%%%%%%%%%
% nội dung này có thể tham khảo giáo trình ĐSTT và HH2
\section{Mục 2}
\subsection{Tiểu mục 1}

\subsection{Tiểu mục 2}

%%%%%%%%%%%%%%%%%%%%%%%%%%%%%%%%%%
\chapter{Một số dạng kiểm định thống kê}

\section{Kiểm đinh Pearson}
\subsection{Hàm phân phối tích luỹ}
\begin{definition}
 Nội dung định nghĩa viết vào đây
\end{definition}

\begin{equation}
    F(x) = P(X<x), \; \forall x \in \mathbb{R}
\end{equation}

\begin{lstlisting}[language=Matlab]
\begin{lstlisting}

function [chi_P, chi_J] = pearson_test_V2(N,n)

    % Random variables
    Z = [1 2 3 4 5];

    % Probability corresponding
    PZ = [0.06 0.23 0.35 0.27 0.09];

    % cho nhich len 0.01 de khong cham dau mut
